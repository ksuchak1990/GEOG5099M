\documentclass[12pt, twoside, a4paper]{article}

\usepackage[sort&compress]{natbib}
\usepackage[nottoc]{tocbibind}

\begin{document}

\title{Research Proposal}
\author{Keiran Suchak --- 200888140}
\maketitle

\section{Title of Research Topic}\label{sec:title}

\section{Research Questions and Statement}\label{sec:questions}

%The research questions and statement should drop out and emerge from the review of the literature and the description of the topic in section b. 
%Essentially this is an evaluation of how well the student has been able to distil the research into a set of solid research questions. 

\section{Description of Research}\label{sec:research_descr}

This section should describe the background to the research topic, provide a review of the relevant literature (methodological and or previous work in this domain) and should point towards research gaps that will be addressed by this thesis. 
Essentially the student is being asked to demonstrate their understanding of the topic area, to show that they can summarise a body of research and to develop an implicit rationale for the need for their proposed research.

A better understanding of how people move around their environment is of great
utility to both academics and policy-makers.
Such knowledge can be made use of in the contexts of urban planning, event
management and emergency response, particularly when considering urban
environments.
Furthermore, this may be of use to those interested in the social issues of
mobility, inclusivity and accessibility of opportunities.

When considering such concepts, investigators often make use of modelling
techniques.
At their most fundamental, models represent our understanding of the system that
we are studying --- an understanding that may not be perfect
\citep{stanislaw1986tests}.
There exist modelling techniques for the simulation of how pedestrians move
around urban spaces.
However, these methods exist largely in isolation of the real-world --- that is
to say that whilst the simulations aim to reflect the real-world, there is no
method by which we can incorporate up-to-date observations into these models to
stop their divergence from reality.

Simulating pedestrian behaviour is often undertaken at a micro-scale, with such
models typically aiming to model at the individual level or on a spatially
fine-grained grid \citep{burstedde2001simulation}.
One of the most prevalent simulation methods in this field is that of
Agent-Based Modelling.
Such methods consist of two key components: agents and environments.
In an Agent-Based Model, we prescribe sets of rules by which individuals
interact with each other and their local environment; as interactions take
place on the micro-scale, we typically observe the emergence of structure at the
macro-scale such as crowding \citep{batty2003discrete} or lane formation
\citep{liu2014agent}.
The evaluation of these rules is often not deterministic and instead introduces
some element of randomness; these stochastic elements aim to emulate the
variability of human behaviour.
The introduction of such randomness in conjunction with an imperfect
understanding of the phenomena at play, however, typically result in simulation
runs diverging from the real system.

In constructing their models, agent-based modellers undertake a development
process that involves model verification, validation and calibration.
We can take these to mean the following:
\begin{itemize}
    \item \textbf{Model verification}: The process of ensuring that the
        implementation is an accurate representation of the model
        \citep{xiang2005verification}.
    \item \textbf{Model validation}: The process of ensuring that the chosen
        model is an accurate representation of the phenomenon that we wish to
        study \citep{crooks2008key}.
    \item \textbf{Model calibration}: The process of searching for model
        parameter values such that we can achieve model validation
        \citep{thiele2014facilitating}.
\end{itemize}
Beyond this, modellers also make efforts to ensure that the initial model
conditions are realistic by setting them based on historical data.

The practices of validation, calibration and setting initial model states based
on historical data are appropriate for offline evaluations such as testing
designs of new buildings or experimenting with different individual behaviours;
however, when aiming to simulate events in real-time, this simple delays the
inevitable divergence of the model from the real system.
Furthermore, model parameters may be transient and thus require to be updated as
time passes and the dynamics evolve.

Given the apparently inevitable divergence of stochastic simulations from the
real system that they aim to model, one may alternatively turn to big data.
Data is now being generated in higher volumes and at greater velocity than ever
before \citep{chen2014big}; however there also exist issues with observational
data from such systems.
Whilst models typically allow us to simulate a whole system, observations are
typically sparse in either time or space (or both); this is to say that we
observations rarely provide complete coverage of events.
We therefore seek a solution whereby we can integrate up-to-date observations
into our models as the models continue to simulate the system.

One of the methods by which we can combine knowledge represented by our model
with observations as they become available is through data assimilation
techniques, which are most commonly used in the field of numerical weather
prediction \citep{kalnay2003atmospheric}.
Such techniques are typically made up of two steps:
\begin{enumerate}
    \item \textbf{Predict}: Run the model forward, estimating the state of the
        system, until new observations become available.
    \item \textbf{Update}: Upon receipt of new observations, combine the model's
        estimate of the system state with the new data.
\end{enumerate}
These steps are repeated iteratively in a cycle.
It is important to note that just as there is error associated with the model,
we also acknowledge that there is observational error associated with the data.
The aim of incorporating the observations into the model is to improve the model
accuracy with respect to the true system.

A large volume of work exists in which such techniques are applied to
meteorological systems where the models are based on differential equations.
Significantly less work exists in which data assimilation methods are applied to
agent-based models --- in particular pedestrian models.
This dissertation therefore aims to expand on the pre-existing work by
implementing a data assimilation scheme known as the Ensemble Kalman Filter in
conjunction with a relatively simple agent-based model of pedestrians crossing a
two-dimensional station from one side to the other.

\section{Description of Methodology and Data}\label{sec:method_descr}

%This section should describe the data and methods that will be used as part of this study. 
%The student should show their understanding of how the data analysis will allows the research aims / questions will be answered. 
%The data should exist, it should be available to the study within the proposed timeframe, indications of communications between data holders and students should be included, and a realistic appraisal of contingency data should be considered.

\section{Timetable of Programme of Work}\label{sec:timetable}

%This should show a realistic plan of work for an August / September submission.

\section{Risk Assessment}\label{sec:risk_ass}

%This should be a general assessment of the possible risks, their likelihood, the strength of their impact and any possible mitigation strategies. 
%Obviously not every eventuality can be considered but general risks to the successful completion of the project should be identified and considered.

\bibliographystyle{agsm}
\bibliography{references}

\end{document}
