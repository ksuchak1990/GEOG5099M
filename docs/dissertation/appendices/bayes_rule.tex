\chapter{Bayes Rule}\label{ch:bayes_rule}

The joint probability --- $P(A, B)$ --- is the probability of event $A$ occurring
and event $B$ occurring.
The conditional probability --- $P(A | B)$ --- is the probability of event $A$
occurring given that event $B$ occurs.
The joint and conditional probabilities are linked by the following relationship:
\begin{equation}\label{eq:ab}
    P(A, B) = P(A) P(B | A).
\end{equation}
Similarly, we have
\begin{equation}\label{eq:ba}
    P(B, A) = P(B) P(A | B).
\end{equation}
Given that the joint probabilities of $A, B$ and $B, A$ are equal:
\begin{equation*}
    P(A, B) = P(B, A),
\end{equation*}
Equations \ref{eq:ab} and \ref{eq:ba} can be put together to give
\begin{equation}\label{eq:pre_bayes}
    P(A) P(B | A) = P(B) P(A | B).
\end{equation}
From this, we can derive the typical form of Bayes Rule by dividing by $P(A)$:
\begin{equation}\label{eq:bayes}
    P(B | A) = \frac{P(A | B) P(B)}{P(A)}.
\end{equation}
