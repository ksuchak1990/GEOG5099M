\chapter{Method}\label{ch:method}

\begin{itemize}
    \item Intro to data assimilation
    \item What is data assimilation?
    \item Where does it come from?
    \item What is the point?
    \item What types of data assimilation are available to us?
    \item Which one are we going to use?
\end{itemize}

The updating of the model state is undertaken on the basis of Bayes Rule:
\begin{equation}
    P(A|B) = \frac{P(B|A) P(A)}{P(B)}
\end{equation}
Bayes Rule is made up of four components:
\begin{enumerate}
    \item $P \left( A \right)$: Prior
    \item $P \left( A|B \right)$: Posterior
    \item $P \left( B|A \right)$: Likelihood
    \item $P \left( B \right)$: Marginal likelihood
\end{enumerate}

Then let's frame our problem notationally --- this is what Bayes theorem looks
like for our situation:
\begin{equation}
    P \left( \mathbf{x} | \mathbf{d} \right) =
       \frac{P \left( \mathbf{d} | \mathbf{x} \right)
             P \left( \mathbf{x} \right)}{P \left( \mathbf{d} \right)} 
\end{equation}
In this case, each of the components are taken to mean
\begin{enumerate}
    \item Prior: The probability of the model state
    \item Posterior: The probability of the model state given the data
    \item Likelihood: The probability of the data given the model state
    \item Marginal likelihood: The probability of the data
\end{enumerate}

\section{Kalman Filter}\label{sec:method:kf}

The Kalman Filter \citep{kalman1960new} is the best linear unbiased estimator,
minimising posterior mean squared error.

\begin{itemize}
    \item What is the Kalman Filter?
    \item How does it work? Sequential data assimilation to update state and
        covariance matrix, weighted based on model forecasts and observations.
    \item When is it good? Optimal solution for linear model and Gaussian
        errors.
    \item When is it bad?
    \item What can we do to improve it?
\end{itemize}

\section{Ensemble Kalman Filter}\label{sec:method:enkf}

Problems with the Kalman Filter:
\begin{itemize}
    \item it assumes Gaussian PDFs
    \item it assumes linear model
    \item Cost of evolving covariance matrix
\end{itemize}

In order address some of these problems, the Ensemble Kalman Filter was
developed \citep{evensen2003ensemble,
evensen2009ensemble}, which acts as an approximation of the Kalman Filter.
This approximation is achieved by using an ensemble of sample state vectors to
represent the state distribution.
As such, the state is represented as follows:
\begin{equation}
    \mathbf{X} = \left[ \mathbf{x}_1, \ldots, \mathbf{x}_N \right]
               = \left[ \mathbf{x}_i \right], \quad \forall i \in (1, N),
\end{equation}
where the state ensemble matrix, $\mathbf{X}$, consists of $N$ state vectors,
$\mathbf{x}_i$.
Similarly, the observations are represented as follows:
\begin{equation}
    \mathbf{D} = \left[ \mathbf{d}_1, \ldots, \mathbf{d}_N \right]
               = \left[ \mathbf{d}_i \right], \quad \forall i \in (1, N),
\end{equation}
with each member of the data ensemble matrix, $\mathbf{D}$, being the sum of the
original observation $\mathbf{d}$, and a random vector, $\mathbf{\epsilon}_i$:
\begin{equation}
    \mathbf{d}_i = \mathbf{d} + \mathbf{\epsilon}, \quad
                   \forall i \in (1, N).
\end{equation}
The random vector is drawn from an unbiased normal distribution:
\begin{equation}
    \mathbf{\epsilon} \sim \mathcal{N} (0, \mathbf{R}).
\end{equation}

\begin{equation}
    \hat{\mathbf{X}} = \mathbf{X} + \mathbf{K}
                       \left(
                       \mathbf{D} - \mathbf{H} \mathbf{X}
                       \right)
\end{equation}

\begin{equation}
    \mathbf{K} = \mathbf{Q} \mathbf{H}^T
                 {\left(
                    \mathbf{H} \mathbf{Q} \mathbf{H}^T
                    + \mathbf{R}
                 \right)} ^ {-1}
\end{equation}

\subsection{Different Types of Ensemble Kalman Filter}\label{sec:method:types}

Talk about the different types of EnKF and the implications for ensemble size
\citep{keller2018comparing}.

\begin{itemize}
    \item Damping: counteract filter divergence
    \item Localisation: reduce the effect of spurious correlations
    \item Hybrid EnKF: Covariance matrix is made up of the weighted sum of the
        usual covariance matrix and a separate static covariance matrix that
        encodes prior underlying knowledge about the system
    \item Dual EnKF: Split the state vector into state and parameters. At
        assimilation: update parameters, recalculate forecast, update state
    \item Normal Score EnKF: Developed to handle non-Gaussian PDFs in EnKF. At
        assimilation: transform state, parameters and measurements into
        Z-scores, perform EnKF update based on transformed values, transform
        back from Z-scores
    \item Iterative EnKF
\end{itemize}
