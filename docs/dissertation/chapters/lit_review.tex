\chapter{Literature Review}\label{ch:lit_rev}

\section{Model Calibration}\label{sec:lit_rev:calibration}

As touched upon in Chapter \ref{ch:intro}, the process of developing an
agent-based model typically involves some form of model calibration.
Model calibration is the procedure of fine-tuning the model that we are using
such that it best fits the particular situation that we are seeking to model
\citep{crooks2012introduction}.
There are a large number of different manners in which we can calibrate
agent-based models \citep{thiele2014facilitating}.

\begin{itemize}
    \item MORE ABOUT MODEL CALIBRATION
\end{itemize}

\section{Data Assimilation}\label{sec:lit_rev:da}

\begin{itemize}
    \item DATA ASSIMILATION IN GENERAL
    \item HOW THIS RELATES TO CALIBRATION
\end{itemize}

The process of data assimilation involves making use of observations along with
prior knowledge (which, in our case, is encoded in a model) to produce
increasingly accurate estimates of variables of interest.
Such a process can be achieved through a Bayesian filtering approach
\citep{grewal1995kalman}.

\subsection{Data Assimilation with Agent-Based Models}\label{sub:lit_rev:da:abm}

\begin{itemize}
    \item \cite{ward2016dynamic} --- model of pedestrians on Briggate, a 1-D
        strip along which pedestrians walk, using enkf for both state and
        parameter calibration.
    \item \cite{wang2013data, wang2015data} --- agents occupying a smart
        environment/building with a view to modelling population density, using
        particle filter.
    \item \cite{rai2013behavior} --- agents occupying a smart office/building,
        using particle filter, extends \cite{wang2013data, wang2015data} by
        incorporating a Hidden Markov Model for behaviour pattern detection.
    \item \cite{wang2017random} --- model of maritime pirates, using random
        finite set based data assimilation instead of kf or pf --- why? does
        this have any relevance?
\end{itemize}

\subsection{Data Assimilation with Cellular Automata}\label{sub:lit_rev:da:ca}

Whilst this dissertation focuses on the application of data assimilation methods
to agent-based models, there also exists a body of work that makes use of the
same methods in conjunction with cellular automata.

\begin{itemize}
    \item WHAT ARE CELLULAR AUTOMATA
    \item HOW DO THEY RELATE TO ABMS?
    \item WHAT IS THE POINT OF INCLUDING THIS?
\end{itemize}

\begin{itemize}
    \item \cite{li2017exploring} --- CA for urban land use, using enkf.
    \item \cite{li2012assimilating} --- CA for urban land use, using enkf.
    \item 
\end{itemize}

